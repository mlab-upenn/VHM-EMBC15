\section{outline}

Intro:\\
- The human heart: a normal EKG, examples of arrythmias.
\\
- Stress that the interpretation of EKGs is complex and equal parts art and science.
\\
- Pacemakers: what is their function, how they are developed
\\
How pacemakers are developed today: code modules, rule-based

Two main decisions when testing:\\
- what input sequences to provide\\
- how to determine correctness of output\\
Ideally:\\
- the input sequences to the device are those that will be generated by heart reacting to the PM's outputs.
\\
- the device correctness is determined by looking at the heart's behavior: is it safe? Safety is a domain-specific concept and must be determined by the physician.

Important: if the heart displays unsafe or undesired behavior, then this means there's room for improvement, but not that the PM should be thrown out. Depending on what our constraints are, some undesired behavior might have to be tolerated if it is a consequence of an even more desirable operation of the PM.

Pacemaker testing today: open-loop. 
Show examples from Medtronic

The shortcomings of open-loop testing:
\\
- only uses input sequences that were thought of by validator, thus might miss inputs provided by the heart and which PM must accommodate. \textbf{how this manifests itslef}: undesired heart behavior even though PM satisfies its spec.
\\
- the measure of correctness is what the PM does, not what the heart does. Going from what we would like the heart to do to what the PM should do is an error-prone, manual, intuitive process.
\textbf{how this manifests itself}: the PM functions correctly (according to its spec) on a heart input that \emph{was} part of the test suite, but the heart does something undesired.
\\
- open-loop tests can still be run in a closed-loop setting via initialization. Only now, we get to see how the heart reacts to the PM and thus determine whether the intended effect is achieved. 


Closed-loop testing connects the heart to the pacemaker. The controlled variables are disturbances like PVC and PAC, and parameters of both pacemaker and heart.
\\
- The input sequences to the PM are provided by the heart, so there's no issue of missing waveforms. we have a systematic way fo exploring the space of inputs (even if we might not get to all of it because of time limitation)
\\
- the correctness is determined by the heart's behavior, as it should.
\\
- can use an actual pacemaker: the HoC 

But we don't have hearts lying around for testing -> heart models.

Specification-guided testing: given the specification that we wish to test, guide the testing process towards behavior that might falsify it, thus producing a concrete counter-example for hte designer to use as feedback.

Experiments:\\
- example of finding a model error \\
- example of undesired behavior

Future:\\
- Enrich the heart model \\
- run on the HoC
