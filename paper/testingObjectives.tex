Device testing has two objectives

\begin{itemize}
	\item \textbf{Test whether the device model, which is used to develop, test and manufacture the device, is indeed correct}.
	This is the classical goal of testing.
	Starting from an initial executable model of the device, the model is refined and made more accurate as it moves through the design process. 
	The model includes the control logic which will ultimately be implemented in software.
	The final model, which forms the basis for manufacturing, must be tested to ensure that it meets its specifications.
	
	The specifications can be open-loop or closed-loop, leading to open-loop or closed-loop testing.
	The distinction between the two is detailed in the next sections, but briefly, an open-loop specifications specifies the output behavior of the pacemaker given an input sequence.
	The heart's behavior is not modeled and the specification does not describe what it should be.
	A closed-loop specification specifies the behavior of either pacemaker, heart, or both, when they are connected as they would be in vivo.
	Thus a closed-loop specification describes the overall closed-loop system, and a heart model is needed for closed-loop testing.
		%
	\item \textbf{Determining the range of parameters and heart conditions under which the device can operate according to certain specifications}.
	A pacemaker is designed to operate differently depending on the environmental conditions. 
	The environment of the pacemaker consists of the heart and the human body in which it operates. 
	For example, \todo[inline]{give example of conditional specification}
	It is not always clear where the limits of the operating conditions are.
	Testing can help identify these boundaries, e.g. by demonstrating that if a certain environmental condition is met, then the pacemaker can no longer fulfill a certain function.
	\todo[inline]{give example}
	Because this behavior is environment-specific, it can only be tested in a closed-loop setting where the heart and external disturbances are modeled.
\end{itemize}