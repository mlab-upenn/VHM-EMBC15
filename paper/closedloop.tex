\section{Closed-loop Testing of Pacemaker}
\label{closedloop}

= SS: Closed-loop.
Table comparing closed-loop with open-loop.
Robustness-guided testing.
%\subsection{Limitations of Current Practice: Open-loop Testing of Pacemakers}
\label{todaysPractice}
\begin{figure}[tb]
	\centering
	\includegraphics[scale=0.2]{placeHolder.pdf}
	\caption{Setup for pacemaker operation: the device (pacemaker), the plant (human heart), external disturbances (PVCs and PACs), and initial states and parameters (refractory periods, various settings).}
	\label{fig:liveSetup}
\end{figure}
Consider Fig.\ref{fig:liveSetup}, where a typical setup of pacemaker operation is presented.
The Device Under Test (DUT) is connected to the Plant (here, a human heart) which it is meant to control.
Together, they are referred to as the \emph{closed loop}, whereas the DUT alone is an \emph{open loop}.
The plant receives inputs from the DUT, but can also be subject to \emph{external disturbances}: in our case, we consider PVCs and PACs to be external disturbances.
The terminology is meant to convey that these disturbances are inputs to the heart that do not originate in the DUT, but elsewhere. 
The origin of these disturbances need not be modeled.  
They are \emph{disturbances} because they are uncontrolled and can prevent the closed loop from achieving its objectives, namely, a safe and reliable heart operation.

The outputs of the pacemaker are denoted by the letter $u$, e.g. $u = [A_{pace}, V_{pace}]$.
Let $U$ be the set of possible $u$ values.
These constitute inputs to the heart, or plant.
The outputs of the heart are denoted by the letter $y$, e.g. $y = [A_{sense}, V_{sense}]$.
Let $Y$ be the set of possible $y$ values.
These constitute inputs to the pacemaker, or DUT.
A finite sequence, or \emph{string}, of $u$ values will be denoted by $\ub$, and the set of \emph{all} finite $u$-sequences will be denoted by $U^*$: $\ub \in U^*$.
Similarly, we define $\yb \in Y^*$ to be a finite $y$-string.
The disturbances are denoted by $w$, their set of possible values by $W$, and strings of disturbance values by $\wb \in W^*$.

A (discrete-time) \emph{open-loop test} $\yb$ for the DUT is a string $\yb = y_1 y_2 \ldots y_n \in Y^*$.
As a result of being fed a test, the DUT will synchronously produce a string $\ub = u_1 \ldots u_n \in U^*$.
A (discrete-time) \emph{closed-loop test} $\wb$ for the closed loop DUT+Plant (pacemaker + heart) is a string $\wb = w_1 w_2 \ldots w_n \in W^*$.
As a result of being fed a test, the closed loop will synchronously produce a string $(\ub,\yb) \in U^* \times Y^*$.
%The plant $\Hc$ may be seen as an input-output map: $\Hc: U^* \rightarrow Y^*$, where $U^* (Y^*)$ is the set of finite strings over the set $U$ ($Y$).

\subsection{Testing objectives}
\label{testingObjectives}
When testing a pacemaker, or any device in general, two crucial decisions must be made:
\begin{itemize}
	\item Choice of inputs: first, what input strings should be provided to the device so its behavior can be observed in reaction to them? 
	Ideally, the answer is: all input sequences that might be produced by the heart to which the device will be connected. 
	Sequences from outside this set are irrelevant, and missing sequences from within this set might cause us to miss some bugs that will manifest themselves during live operation.
	In what follows, we use $A_\Hc$ to denote the set of strings that might be produced by the heart when connected to the pacemaker in a closed loop as shown in Fig.\ref{fig:liveSetup}. 
	Of course, $A_\Hc \subset Y^*$.
	\item Criterion of correctness: secondly, how do we determine that a particular operation of the device is correct? 
	In other words, what is the specification according to which the device is being tested?
	Ideally, the answer is: the specification will describe the safe, acceptable operation of the heart. 
	I.e. the device is correct if the heart it is controlling does not produce unsafe or undesirable behavior.
	Because the heart is a complex system and its `acceptable' behavior is dependent on its structural characteristics and current condition, such specifications are hard to describe. 
	Nonetheless, this is the standard for what a specification should be for the DUT. 
	A DUT that violates this specification may or may not be modified, depending on the source of the violation. 
	The task of interpreting test outcomes and acting upon them is left to the designer and physician.	
	The validation engineer's task is to produce a trace that violates the specification, or confirm that the DUT does not violate it with high confidence.
\end{itemize}
Note that we distinguish between unsafe heart behavior and undesirable heart behavior: unsafe behavior is not tolerated under any circumstances.\todo[inline]{give example}
Undesirable behavior may not be ideal, but it might be acceptable under certain conditions. 
\todo[inline]{give example}
It is important to understand the circumstances under which undesirable behavior occurs, and then decide whether this is acceptable or not.
In Section \ref{experiments}, we illustrate such a case.
The goal of testing is to establish that the heart's behavior is never unsafe, and establish conditions under which it might be undesirable.

\subsection{Open-loop testing}
\label{openLoopTesting}
\begin{figure}[tb]
	\centering
	\includegraphics[scale=0.2]{placeHolder.pdf}
	\caption{Open loop testing setup: the device (pacemaker) is fed with pre-programmed open-loop tests.}
	\label{fig:OLtestingSetup}
\end{figure}
Today, pacemakers are tested in an open loop \todo[inline]{cite}.
That is, the pacemaker is not connected to the heart, as shown in Fig.\ref{fig:OLtestingSetup}.
Rather, the validation engineer provides the DUT with $N$ open-loop tests
\[A_{OL} = \{\yb_k, k=1,\ldots,N\}\]
E.g., each test is a string of $A_{sense}$ and $V_{sense}$ values. The validator observes the DUT's output $\ub$ to determine correctness.
The tests $\yb_k$ are meant to mimic some possible signals produced by the heart and are designed in consultation with physicians.
They are also \emph{the same tests that were used when designing the pacemaker}.
\todo[inline]{check that last statement}
What the pacemaker's reaction should be is decided based on a diagnosis of the heart signals $\yb_k$:
that is, the physician diagnoses what might cause the heart to produce a certain $\yb$, and decides accordingly what the pacemaker should do.
The heart is thus only implicitly involved in the testing process. 
How does this compare with the ideal state of affairs outlined in Section \ref{testingObjectives}?
\begin{itemize}
	\item Choice of inputs:	There is no guarantee that the validator's tests satisfy $A_{OL} = A_
	Hc$. 
	Thus testing might miss valid test cases ($\yb \in A_\Hc \setminus A_{OL}$), or include irrelevant ones ($\yb \in A_{OL} \setminus A_\Hc$).
	Moreover, there is no automatic way of producing the tests: rather, it is a manual and error-prone process.
	\item Criterion of correctness: because the heart does not figure in open-loop testing, correctness is decided based on the pacemaker's reactions. 
	What these mean for the heart is only implicit or assumed, but is not actually tested. 
	While there is no substitute for the physician's expertise, the fact that they need to diagnose every test means there will be inevitable errors or lapses.
\end{itemize}

In the next section, we present a closed-loop testing setup which addresses both shortcomings of open-loop testing.


\begin{figure}[!t]
	\centering
	\includegraphics[scale=0.2]{placeHolder.pdf}
	\caption{\small Requirement-guided closed-loop testing methodology}
	\label{fig:reqGuidedTesting}
\end{figure} 

\textbf{Controlled inputs}.
In our setup, the testing algorithm will generate PVC waveforms to mimic the abnormal depolarizations of the ventricles that occur in a heart with arrythmia. 
These are used by the tester to try and cause the closed loop to manifest unsafe or undesirable heart conditions behavior. 
The constraint on the waveforms generated by the tester is that there should be at least 400ms between consecutive PVC impulses.

\textbf{Closed-loop setup}.
The closed-loop setup is the same as the setup for live operation shown in Fig.~\ref{fig:liveSetup}, with the addition of the tester at the PVC inputs to the heart.
That is, the pacemaker is connected to the heart, and the tester controls the external PVC disturbances.
The inputs to the pacemaker are provided by the heart model, and don't need to be pre-programmed by the validator.
This is shown in Fig.~\ref{fig:reqGuidedTesting}.

\textbf{Advantages over open-loop testing}.
The advantages of closed-loop testing over open-loop testing are summarized in Table \ref{table:CLoverOL}
How does closed-loop testing compare to the ideal state of affairs depicted in Section~\ref{testingObjectives}?
\begin{itemize}
	\item Choice of inputs: because a VHM provides the inputs to the pacemaker, we know that $A_{CL} \subset A_\Hc$, where $A_{CL}$ is the set of strings $\yb$ fed to the pacemaker in a closed-loop setup. 
	Thus, there is no risk of testing irrelevant scenarios\footnote{Of course, this ultimatley depends on the quality of the VHM.}
	This leaves open, of course, the question of how the tester chooses the disturbance inputs $\wb$. The answer is provided in Section \ref{testingMethodology}, but briefly, the $\wb$ choice is automatic and guided by the specification that we are testing, and is asymptotically complete (i.e. in the limit, $A_{CL} = A_\Hc$).
	\item Criterion of correctness: because we have a VHM, we can express correctness \emph{as a property of the heart's behavior}, and not of the pacemaker alone. 
	Thus we can evaluate what truly matters: is the heart (as modeled by the VHM) displaying unsafe or undesirable behavior?
\end{itemize}

It must be stressed that the interpretation of closed-loop testing results (bug or feature of the pacemaker?) depends on the specific violating behaviors that are found.
Some will be determined to be bugs. 
Others will be determined to be undesirable but necessary behaviors.
This is because specifying `safe' behavior is difficult for something as complex as the heart.

\textbf{The specification-guided tester}.



\subsection{Testing Methodology}
\label{testingMethodology}

\subsection{Testing}
\label{testingObjectives}
Device testing has two objectives

\begin{itemize}
	\item \textbf{Test whether the device is indeed correct}.
	This is the classical goal of testing.
	Correctness is defined by a set of specifications that must be met.	
	The specifications can be open-loop or closed-loop, leading to open-loop or closed-loop testing.
	The distinction between the two is detailed in the next sections, but briefly, an open-loop specification specifies the output behavior of the pacemaker given an input sequence.
	Such a specification does not refer to the heart's behavior.
	E.g., such a specification could stipulate that \todo[inline]{add one from boston scientific's manual}
	The heart's behavior is not modeled in open-loop testing and the specification does not describe what it should be.
	
	A closed-loop specification specifies the behavior of either pacemaker, heart, or both, when they are connected as they would be in vivo.
	Thus a closed-loop specification describes the overall closed-loop system, and a heart model is needed for closed-loop testing.
	It can be argued that what ultimately matters is the closed-loop behavior, and that open-loop testing is a proxy for testing closed-loop behavior.
		%
	\item \textbf{Determining the range of parameters and heart conditions under which the device can operate according to certain specifications}.
	A pacemaker is designed to operate differently depending on the environmental conditions. 
	The environment of the pacemaker consists of the heart and the human body in which it operates. 
	For example, \todo[inline]{give example of conditional specification}
	It is not always clear where the limits of the operating conditions are.
	Testing can help identify these boundaries, e.g. by demonstrating that if a certain environmental condition is met, then the pacemaker can no longer fulfill a certain function.
	\todo[inline]{give example}
	Because this behavior is environment-specific, it can only be tested in a closed-loop setting where the heart and external disturbances are modeled.
\end{itemize}
