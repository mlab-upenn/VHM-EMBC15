\section{Introduction}
\label{introduction}

Medical devices like implantable pacemakers are designed to diagnose and improve undesired physiological conditions. The capability to affect the physiological conditions of the patient makes the safety of the devices an essential consideration. The software component of these devices are getting increasingly complex, inevitably leading to more safety violations. From 1996-2006, the percentage of software-related causes in medical device recalls have grown from 10\% to 21\%~\cite{recalls}. During the first half of 2010, the US Food and Drug Administration (FDA) issued 23 recalls of defective devices, all of which can cause serious adverse health consequences or death. At least six of the recalls were caused by software defects~\cite{killedbycode}. 

To ensure the safety and efficacy of a device software, we first need to ensure that the device software exhibit the input-output relationship as designed, which are captured in the \emph{Software Specifications}. However, we also have to ensure that the device software is capable of improve the physiological conditions as expected, which are captured in the \emph{Physiological Requirements}. The procedure evaluating the conformance between the device software and these two design documents are referred to as \emph{Verification}. The US FDA requires device manufacturers to submit \emph{sufficient evidence} regarding the safety and efficacy of the devices before they can be released to the market. The specifications of the device software is verified using \emph{open-loop testing} in which the devices are given input signals and their outputs are compared with expected outputs according to the design. However, open-loop testing is not effective in finding safety violations which have deeper executions and corresponding to real physiological context. The physiological requirements have to be verified in closed-loop testing in which the devices are tested within its physiological context. Currently the physiological requirements are mostly verified in terms of  \emph{clinical trials} in which the devices are implanted in selected groups of patients and monitored over certain period of time. The limitations of clinical trials are both extreme high cost and limited sample size of patient groups.

Model-based design enables closed-loop verification at earlier design stage. The device (or corresponding software or even mathematical model) interacts with a physiological model which can simulate signals into the device and respond to device outputs. In this work, we use a physiological heart model 

